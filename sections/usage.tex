\section{Usage}

\subsection{Running the LaTeX document}

\subsubsection{Requirements}

You'll need to have \LaTeX  installed. Also, you'll need \textbf{python} with \textbf{Pygments}:

\inputminted{python}{src/installPygments.py}

This ensures, that you can highlight code.
\begin{minted}{python}
def test(args):
	for elem in args:
		print(elem)
		if "something" in elem:
			out = elem
	return(out)
\end{minted}

\subsection{Compiling the document}

You can compile the \LaTeX  document using the provided \textbf{latexrun} package or the standard \textbf{pdflatex}. The latter will not provide very useful error-prompts and the prior (\textbf{latexrun}) is suggested as an alternative. 

\begin{minted}{shell}
#Using pdflatex
pdflatex --shell-escape master.tex

#Using latexrun
./latexrun --latex-args "\-shell-escape" -O . -o report.pdf master.tex
\end{minted}

\subsection{Using the .sty file}

\subsubsection{Code highlighting}
The \texttt{.sty}-file provides a package for code highlights, which can be embedded in your \LaTeX  documentation. Use as in this file. You can either do \textbf{inline code} or use from \textbf{file}. 

You can alter the highlight-colors by changing the \textbf{usemintedstyle{}}-argument in your \textbf{master.tex}-file.

\subsubsection{Setting up your project}
In your \textbf{master.tex}-file simply put your project name into: \\
 \begin{verbatim}\setProject{<yourproject>}\end{verbatim} 
... and your name into: \\
 \begin{verbatim}\setAuthor{<yourname>}\end{verbatim}
Now, everything will be added to the header as well as title.





